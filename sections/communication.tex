%--------------------------------------------------------
\section{Client-Server Communication} \label{sec:communication}
%--------------------------------------------------------

In this section, we create a communication model between the server and clients
based around the four Coexist API methods listed in Section~\ref{sec:api}. In
Coexist, every client fully replicates the content of the central database, so
the transfer of information ends up resembling Figure~\ref{fig:mirror}.


\begin{figure}[h!]
\centering
\begin{tikzpicture}[node distance = 2cm, auto, scale=.7, every node/.style={transform shape}]
  % Place nodes
  \node [block] (client) {client};
  \node [block, right of=client, node distance=5cm] (server) {server};
  \node [db, below of=client, node distance=4cm, align=center] (database) {local \\ database};

  % Draw edges
  % Need to align in order to line break
  \path [line] (client) -- node {udpate} (server);
  \path [line] (server) |- node {sync} (database);
  \path [line] (database) -- node [align=left] {query} (client);
\end{tikzpicture}
\caption{The flow of information in Coexist. Clients send updates to the server
and store the contents of requests in their local database. Clients never modify
their local database.}
\label{fig:mirror}
\end{figure}

All clients in Coexist send requests to add and update data to the server
through \create. When the server confirms that the request was successful, the
client is free to initiate a sync, which will bring down all updated rows via
the methods described in Section~\ref{sec:replication}. All read operations are
performed on the local database and writes only occur through syncs. This
guarantees that clients will never have more information than the server so
users can rest assured that any data they can view must also be on the central
server and all other clients. From here, we can discuss the details of the four
API methods, starting with \schema in Figure~\ref{fig:request_schema}. 


%--------------------------------------------------------
\subsection{Request and response structure}  \label{sec:}

In Coexist, client-server communication happens through four API methods.  This
section describes the parameters that clients are required to send in each
request and the responses they can expect from the server. In figures, requests
are shown as query strings with JSON values and responses are JSON objects. The
first API method is \schema in Figure~\ref{fig:request_schema}.

\begin{figure}[h!]
\begin{subfigure}[b]{0.23\textwidth}
\begin{lstlisting}[language=json]
db="sqlite"
\end{lstlisting}
\caption{Client HTTP request.}
\label{fig:gull}
\end{subfigure}%
\begin{subfigure}[b]{0.27\textwidth}
\begin{lstlisting}[language=json]
{
  "status":200,
  "message":"OK",
  "version":1,
  "sql":["..."]
}
\end{lstlisting}
\caption{Server JSON response.}
\label{fig:tiger}
\end{subfigure}
\caption{A sample client request to \schema, where the left side is the client's
request parameters and the right is the server's response.}
\label{fig:request_schema} \end{figure}

The purpose of \schema is to allow the client to retrieve schema information
from the server. The only information that the server needs from a client in
order to fulfil its request is the type of database that it uses, so the only
required parameter is \var{db}. In this example, the client uses a SQLite
database, so it passes a value of \var{sqlite} for the \var{db} parameter.
Currently, Coexist doesn't take advantage of any non-standard SQL features
\var{db} isn't actually used, but that may change in the future. The \metamodel
request is identical to the \schema request, but its response returns a JSON
based metamodel, described in Section~\ref{sec:metamodel}.

The response from the server contains an obligatory \var{status} and
\var{message} parameter, followed by the current version of the server side
database schema (used in requests to \sync), and a \var{sql} parameter
containing the raw SQL that the client can use to construct its local database.
The \sync request shown in Figure~\ref{fig:request_sync} is more complicated.


\begin{figure}[h!]
\begin{subfigure}[b]{0.23\textwidth}
\begin{lstlisting}[language=json]
version=1&
tables=
  [
  "BodyParts",
  "Exercises",
  "Measurements",
  "Sets"
  ]&
mod_ts=
  [
  "2013-02-1104:21:06",
  "2013-04-20 22:33:24",
  "2013-03-14 05:51:14",
  "2013-04-20 22:42:58"
  ]
\end{lstlisting}
\caption{Client HTTP request.}
%\label{fig:gull}
\end{subfigure}%
\begin{subfigure}[b]{0.27\textwidth}
\begin{lstlisting}[language=json]
{
  "status":200,
  "message":"OK",
  "tables":
    {
      "BodyParts":[],
      "Exercises":[
        {
          "exercise":"bb curls",
          "mod_ts":"2013-04-21 04:32:18",
          "deleted":"0"
        }
      ],
      "Measurements":[],
      "Sets":[]
    }
}
\end{lstlisting}
\caption{Server JSON response.}
\label{fig:request_sync_response}
\end{subfigure}
\caption{A sample client request to \sync, where the left side is the client's
request parameters and the right is the server's response.}
\label{fig:request_sync} 
\end{figure}

Request structure is taken right from Section~\ref{sec:replication}. Each
request to \sync has a \var{version}, \var{tables}, and \var{mod\_ts}
parameter. The \var{version} parameter should contain the current client
version, which it will learn through a call to \schema. This version aligns
with the schema being used on the server. If the client version does not match
the server version then the client is out of date and needs to restart the
its life cycle, shown in Figure~\ref{fig:client_states}.

The \var{tables} parameter takes a JSON array of table names in no particular
order. These names will be the tables of the client's local database. In this
example, the client's local database has tables named Body Parts, Exercises,
Measurements, and Sets. This is closely related to the \var{mod\_ts} parameter
which takes a JSON array of modification timestamps where the $i^{th}$ timestamp is
the most recent modification timestamp for $i^{th}$ table in \var{tables}

The response from the server contains the \var{tables} parameter, which is a
JSON object where each attribute is named after a table. Each table attribute
contains an array of objects that represent missing rows that need to be added
to the client based on the timestamps it sent. In this example, the
\var{Exercises} table was missing one row with columns \var{exercise},
\var{mod\_ts}, and \var{deleted}. Currently, the column names are contained in
every object, which doubles the size of the request. In the future, the values
will not be objects with table name keys, they will just be arrays of values.
This means that \var{Exercises} key in Figure~\ref{fig:request_sync_response}
would have a value of \var{[["bb curls","2013-04-21 04:32:18","0"]]}, and list
the column names outside of the array. The \sync and \create requests are the
bread and butter of Coexist clients, making up the majority of functionality and
percentage of requests. Figure~\ref{fig:request_create} contains a sample
request to \create.

\begin{figure}[h!]
\begin{subfigure}[b]{0.23\textwidth}
\begin{lstlisting}[language=json]
version=1&
sync=
  {
  "tables":
    {
      "Exercises":[
        {
          "Exercise":
          "bb curls"
        }
      ]
    }
\end{lstlisting}
\caption{Client HTTP request.}
\label{fig:gull}
\end{subfigure}%
\begin{subfigure}[b]{0.27\textwidth}
\begin{lstlisting}[language=json]
{
  "status":200,
  "message":"OK"
}
\end{lstlisting}
\caption{Server JSON response.}
\label{fig:tiger}
\end{subfigure}
\caption{A sample client request to \create, where the left side is the client's
request parameters and the right is the server's response.}
\label{fig:request_create} 
\end{figure}


Requests to \create look very similar to the server responses from \sync. In
general, clients send a \var{version} and \var{sync} parameter, where \var{sync}
contains an object that maps table names to arrays of objects that should be added
to the server side database. If there are no issues adding the row, then the
server replies with a 200 status code and the client is free to call \sync to
bring the new rows into its local storage. 

The examples in this section so far have assumed that requests are successful.
In reality, there are a few scenarios where clients can run into conflicts,
namely when
\begin{inparaenum}
\item making requests to \sync and \create with a mismatching version and
\item making requests to \create that violate key constraints. 
\end{inparaenum}
The latter was addressed in Figure~\ref{fig:create} in
Section~\ref{sec:replication}. The former is addressed with status codes in
server responses based on HTTP status codes. In each of the examples so far, the server's response has
included a status of 200. Figure~\ref{fig:protocol} shows the dialog between a
client and the server when the server responds with a 409 status code because the
client's \var{version} parameter is 1, when the server's is 2.



\begin{figure}[h!]
\centering
\begin{tikzpicture}[node distance = 2cm, auto, scale=.7, every node/.style={transform shape}]
  % Place nodes
  \node [block] (client) {client};
  \node [block, right of=client, node distance=6cm] (server) {server};
  
  \node [dot, below of=client, node distance=2.5cm, label=180:1] (c1) {};
  \node [dot, below of=server, node distance=2.5cm] (s1) {};
  \path [line] (c1) -- node [above, align=left] {
    \tt\textbf{GET /api/sync} \\ 
    \tt version:1 \\ 
    \tt tables:[t1,t2] \\ 
    \tt mod\_ts:[33,12]}   (s1);
  \path [edge] (client) -- (c1);
  \path [edge] (server) -- (s1);

  \node [dot, below of=c1, node distance=1cm, label=180:2] (c2) {};
  \node [dot, below of=s1, node distance=1cm] (s2) {};
  \path [line] (s2) -- node [above] {\tt\textbf{409 CONFLICT}} (c2);
  \path [edge] (c2) -- (c1);
  \path [edge] (s2) -- (s1);

  \node [dot, below of=c2, node distance=1.5cm, label=180:3] (c3) {};
  \node [dot, below of=s2, node distance=1.5cm] (s3) {};
  \path [line] (c3) -- node [above, align=left] {
    \tt\textbf{GET /api/schema} \\ 
    \tt db:SQLite            } (s3);
  \path [edge] (c2) -- (c3);
  \path [edge] (s2) -- (s3);

  \node [dot, below of=c3, node distance=2cm, label=180:4] (c4) {};
  \node [dot, below of=s3, node distance=2cm] (s4) {};
  \path [line] (s4) -- node [above, align=left] {
    \tt\textbf{200 OK} \\
    \tt\{version:2 \\
    \tt~SQL:[CREATE TABLE...]\} } (c4);
  \path [edge] (c3) -- (c4);
  \path [edge] (s3) -- (s4);
    
  \node [dot, below of=c4, node distance=2cm, label=180:5] (c5) {};
  \node [dot, below of=s4, node distance=2cm] (s5) {};
  \path [line] (c5) -- node [above, align=left] {
    \tt\textbf{GET /api/sync} \\ 
    \tt version:2 \\ 
    \tt tables:[t1,t2,t3] \\ 
    \tt mod\_ts:[0,0,0]}   (s5);

  \path [edge] (c5) -- (c4);
  \path [edge] (s4) -- (s5);


  \node [dot, below of=c5, node distance=2cm, label=180:6] (c6) {};
  \node [dot, below of=s5, node distance=2cm] (s6) {};
  \path [line] (s6) -- node [above, align=left] {
    \tt\textbf{200 OK} \\
    \tt \{t1:[(..),(..)] \\
    \tt~t2:[] \\
    \tt~t3:[] \}} (c6);
  \path [edge] (c6) -- (c5);
  \path [edge] (s6) -- (s5);

\end{tikzpicture}
\caption{The resulting dialog between a client and the server caused by a client
request to \sync that supplied a \var{version} parameter that did not match the
server's, indicating that the client needs to update its schema.}
\label{fig:protocol}
\end{figure}

The actions can be described as follows. 

\begin{enumerate}

\item The client sends a GET request to /api/sync on the server, including the database version it is using, an array of table names and an array of maximum modification timestamps for those tables (found in a metacolumn).

\item The server responds with a 409 error message because it is currently using version 2 of the database, while the client is using version 1.

\item The client sends a GET request to /api/schema on the server to get the current schema that the server side database is using. It requests the SQL in SQLite syntax.

\item The server responds with a 200 OK code. The body of the response is a JSON object containing the version of the server's database and the SQL statements that need to be executed in SQLite to build it. The constraints do not need to be sent to the client, only the table declarations (and possibly views). Since the client never updates its own database through SQL, the constraints are not necessary. 

\item The client now has enough information to re-create its database. Since there is nothing on the client that isn't also on the server, the client can simply replace the contents of its database with this new schema. A call to /api/sync is repeated with the new schema. 

\item There is no version conflict; the server responds with a 200 OK code. The body of the response is a JSON object that maps table names to arrays containing all of the tuples that the client is missing. 

\end{enumerate}


The dialog would look slightly different for an out of sync client attempting to
make server changes. In figure~\ref{fig:protocol}, the client supplies the
server with all of the tables and their most recent update timestamps. If the
client was to send this information with each API request then the server could
similarly notify the client when they appear out of date; but timestamps and
table names can be verbose, it would better to send a representative checksum in
their place. Figure~\ref{fig:signature} represents the sha1 checksum that can be
sent in API requests as a \textit{signature} to validate the client; $T$
represents the concatenation of all table names, and $M$ represents the
concatenation of all maximum modification timestamps. The first timestamp should
correspond to the first table, and so on.

\begin{figure}[h!]
\[
sha1(version + T + M + username + sha1(password))
\]
\caption{API request signature.}
\label{fig:signature}
\end{figure}

So for a schema that consisted of the tables $Students$ and $Courses$, with
maximum modification timestamps of $2012-12-01 16:18:31$ and
$2012-12-01 16:18:43$ respectively, under version $1$ of the database, with a
user name of $tim$ and a password of $password$, the signature would be
equal to \hbox{$9d64e5fbba865b491adeeeead91666df8d7cb1ec$}.

%TODO recalculate checksum


