%--------------------------------------------------------
\section{Client-Server Communication} \label{sec:communication}
%--------------------------------------------------------



%Many mobile applications are supplemented by a local database (typically
%SQLite). Some applications are stand alone and use these databases primarily as
%persistent storage; the Networkmaine inventory application uses its local
%database to interface with the remote back end MySQL server that it depends
%heavily on. There were three methods for this interfacing that represent
%different degrees of caching, from none, to selective, to complete.


%Before deciding upon a

This section will explore some of the methods that the client can use to
interface with the server, their advantages and disadvantages, and explain why
the method labeled as the \textit{mirror method} is the appropriate choice. The
following subsections will give details on the implementation of this method and
its implications on the database schema; particularly, how normalized it should
be and how to handle deletions. In all diagrams in this section, the arrows
indicate the flow of data and the labels indicate the actions that cause them.


%-------------------------------------------------------
\subsubsection{Window method}

In the window method, the data that a user sees at any given time is a window
into the back end server; all of the data came from the server and none of it is
going to persist beyond this screen. The major advantage to this method is its
ease of implementation. This would be a suitable fit for a small weather
application; in fact, it is the approach that my own weather application ,
QtWeather, uses.  

%TODO add some more analysis details

\begin{figure}[h!]
\centering
\begin{tikzpicture}[node distance = 2cm, auto, scale=.7, every node/.style={transform shape}]
  % Place nodes
  \node [block] (client) {client};
  \node [block, right of=client, node distance=5cm] (server) {server};
  % Draw edges
  \path [line] (server.165) -- node [above] {query} (client.15);
  \path [line] (client.345) -- node [below] {udpate} (server.195);
\end{tikzpicture}
\caption{Window method}
\label{fig:window}
\end{figure}


%-------------------------------------------------------
\subsubsection{Cache method}

Treating the client as a window requires that we don't have large amounts of
data constantly being retrieved from the server. If this is not the case, then
we may want to identify some data (or queries) that are commonly used in order
to cache (and optionally sync) on the client's local database. This would be
suitable for a mobile social networking client. Take a typical Facebook
application for example; it would be advantageous to cache a users top five
commonly used groups as well as their friend list because this information is
supposedly accessed often and is relatively static. In contrast, caching the
Facebook news feed would probably result in a lot of wasted space because of how
fast it changes and how rarely old news feed entries are viewed. 



\begin{figure}[h!]
\centering
\begin{tikzpicture}[node distance = 2cm, auto, scale=.7, every node/.style={transform shape}]
  % Place nodes
  \node [block] (client) {client};
  \node [block, right of=client, node distance=5cm] (server) {server};
  \node [db, below of=client, node distance=4cm, align=center] (database) {local \\ database};

  % Draw edges
  % Need to align in order to line break
  \path [line] (server.165) -- node [above, align=center] {query \\ un-cachable data} (client.15);
  \path [line] (client.345) -- node [below] {update} (server.195);
  \path [line] (server) |- node {sync cache} (database);
  \path [line] (database) -- node [align=left] {query frequent \\ and static data} (client);
\end{tikzpicture}
\caption{Cache method}
\label{fig:cache}
\end{figure}


%-------------------------------------------------------
\subsubsection{Mirror method}
The window method has the client retrieve all data from the server all of the
time, the cache method identifies some useful data to store and then caches it
locally; the mirror method is the next natural step in this progression: cache
the entire database. There are some immediate concerns with this approach, most
prominent of which concerns the size of the database potentially being much
larger than the what we can reasonably expect to store on the client. The
advantages to this method are faster query times and less reliance on the
network. In fact, this is the only method that would enable a fully readable
database off line.

\begin{figure}[h!]
\centering
\begin{tikzpicture}[node distance = 2cm, auto, scale=.7, every node/.style={transform shape}]
  % Place nodes
  \node [block] (client) {client};
  \node [block, right of=client, node distance=5cm] (server) {server};
  \node [db, below of=client, node distance=4cm, align=center] (database) {local \\ database};

  % Draw edges
  % Need to align in order to line break
  \path [line] (client) -- node {udpate} (server);
  \path [line] (server) |- node {mirror} (database);
  \path [line] (database) -- node [align=left] {query} (client);
\end{tikzpicture}
\caption{Mirror method}
\label{fig:mirror}
\end{figure}

The Networkmaine inventory database contains information for approximately six
thousand devices, location, purchase order and rma information for batches of
devices; the size of the MySQL database dump is 1.5 megabytes. In addition, the
database will not grow quickly. The size of the database makes the mirror method
very attractive: if Networkmaine added two thousand units a year to its
deployment then in twenty years the size of the database would be no less than
12 megabytes; still reasonable to store on the mobile clients we see available
today (probably more so on the mobile clients in twenty years, if we still use
them). Realistically, there won't be more than a few hundred devices added to
the inventory each year, at most. It is for these reasons that the mirror method
is preferable for this project. 





%--------------------------------------------------------
\subsection{Database Replication} \label{sec:db}


It is important to ensure that the client can never update the server if the
client is out of sync, and the client must always have a way of re-syncing
itself.  This can be achieved with two simple API functions, \texttt{/api/sync}
and \texttt{/api/schema}; where one function requests missing information and
another requests schema information. This design accounts for the database
schema being upgraded on the server, and clients attempting to make changes to
the database server when out of sync. Figure~\ref{fig:protocol} depicts the
network dialog between a client and server where the client attempts to sync
itself when it is using and old version of the database. 




\begin{figure}[h!]
\centering
\begin{tikzpicture}[node distance = 2cm, auto, scale=.7, every node/.style={transform shape}]
  % Place nodes
  \node [block] (client) {client};
  \node [block, right of=client, node distance=6cm] (server) {server};
  
  \node [dot, below of=client, node distance=2.5cm, label=180:1] (c1) {};
  \node [dot, below of=server, node distance=2.5cm] (s1) {};
  \path [line] (c1) -- node [above, align=left] {
    \tt\textbf{GET /api/sync} \\ 
    \tt version:1 \\ 
    \tt tables:[t1,t2] \\ 
    \tt mod\_ts:[33,12]}   (s1);
  \path [edge] (client) -- (c1);
  \path [edge] (server) -- (s1);

  \node [dot, below of=c1, node distance=1cm, label=180:2] (c2) {};
  \node [dot, below of=s1, node distance=1cm] (s2) {};
  \path [line] (s2) -- node [above] {\tt\textbf{409 CONFLICT}} (c2);
  \path [edge] (c2) -- (c1);
  \path [edge] (s2) -- (s1);

  \node [dot, below of=c2, node distance=1.5cm, label=180:3] (c3) {};
  \node [dot, below of=s2, node distance=1.5cm] (s3) {};
  \path [line] (c3) -- node [above, align=left] {
    \tt\textbf{GET /api/schema} \\ 
    \tt db:SQLite            } (s3);
  \path [edge] (c2) -- (c3);
  \path [edge] (s2) -- (s3);

  \node [dot, below of=c3, node distance=2cm, label=180:4] (c4) {};
  \node [dot, below of=s3, node distance=2cm] (s4) {};
  \path [line] (s4) -- node [above, align=left] {
    \tt\textbf{200 OK} \\
    \tt\{version:2 \\
    \tt~SQL:[CREATE TABLE...]\} } (c4);
  \path [edge] (c3) -- (c4);
  \path [edge] (s3) -- (s4);
    
  \node [dot, below of=c4, node distance=2cm, label=180:5] (c5) {};
  \node [dot, below of=s4, node distance=2cm] (s5) {};
  \path [line] (c5) -- node [above, align=left] {
    \tt\textbf{GET /api/sync} \\ 
    \tt version:2 \\ 
    \tt tables:[t1,t2,t3] \\ 
    \tt mod\_ts:[0,0,0]}   (s5);

  \path [edge] (c5) -- (c4);
  \path [edge] (s4) -- (s5);


  \node [dot, below of=c5, node distance=2cm, label=180:6] (c6) {};
  \node [dot, below of=s5, node distance=2cm] (s6) {};
  \path [line] (s6) -- node [above, align=left] {
    \tt\textbf{200 OK} \\
    \tt \{t1:[(..),(..)] \\
    \tt~t2:[] \\
    \tt~t3:[] \}} (c6);
  \path [edge] (c6) -- (c5);
  \path [edge] (s6) -- (s5);

\end{tikzpicture}
\caption{Synchronization protocol}
\label{fig:protocol}
\end{figure}

The actions can be described as follows. 

\begin{enumerate}

\item The client sends a GET request to /api/sync on the server, including the database version it is using, an array of table names and an array of maximum modification timestamps for those tables (found in a metacolumn).

\item The server responds with a 409 error message because it is currently using version 2 of the database, while the client is using version 1.

\item The client sends a GET request to /api/schema on the server to get the current schema that the server side database is using. It requests the SQL in SQLite syntax.

\item The server responds with a 200 OK code. The body of the response is a JSON object containing the version of the server's database and the SQL statements that need to be executed in SQLite to build it. The constraints do not need to be sent to the client, only the table declarations (and possibly views). Since the client never updates its own database through SQL, the constraints are not necessary. 

\item The client now has enough information to re-create its database. Since there is nothing on the client that isn't also on the server, the client can simply replace the contents of its database with this new schema. A call to /api/sync is repeated with the new schema. 

\item There is no version conflict; the server responds with a 200 OK code. The body of the response is a JSON object that maps table names to arrays containing all of the tuples that the client is missing. 

\end{enumerate}


The dialog would look slightly different for an out of sync client attempting to
make server changes. In figure~\ref{fig:protocol}, the client supplies the
server with all of the tables and their most recent update timestamps. If the
client was to send this information with each API request then the server could
similarly notify the client when they appear out of date; but timestamps and
table names can be verbose, it would better to send a representative checksum in
their place. Figure~\ref{fig:signature} represents the sha1 checksum that can be
sent in API requests as a \textit{signature} to validate the client; $T$
represents the concatenation of all table names, and $M$ represents the
concatenation of all maximum modification timestamps. The first timestamp should
correspond to the first table, and so on.

\begin{figure}[h!]
\[
sha1(version + T + M + username + sha1(password))
\]
\caption{API request signature.}
\label{fig:signature}
\end{figure}

So for a schema that consisted of the tables $Students$ and $Courses$, with
maximum modification timestamps\footnote{These timestamps are the same ones
depicted in figure~\ref{fig:student_init}} of $2012-12-01 16:18:31$ and
$2012-12-01 16:18:43$ respectively, under version $1$ of the database, with a
user name of $anthonyn$ and a password of $password$, the signature would be
equal to \hbox{$9d64e5fbba865b491adeeeead91666df8d7cb1ec$}.

%begin{figure}[h!]
%sha1("1StudentsCourses2012-12-01 16:\\
%8:312012-12-01 16:18:43anthonyn5baa6\\
%e4c9b93f3f0682250b6cf8331b7ee68fd8") \\
% 9d64e5fbba865b491adeeeead91666df8d7cb1ec$
%caption{My Listing}
%label{fig:list}
%end{figure}


