%--------------------------------------------------------
\section{The Server and API} \label{sec:api}
%--------------------------------------------------------

In the Coexist client-server model, the clients communicate with the server
through an HTTP based API. This section will specify the details of the
communication between the server and clients, starting with the declaration of
necessary API methods in Section~\ref{sec:api_methods}.


%--------------------------------------------------------
\subsection{API Methods}  \label{sec:api_methods}

% declaring the api methods, not implementing


% maybe introduce the data replication stuff before this point. 
The API can be designed by taking the client's needs into
consideration. In Coexist, every client has a local copy of their central
database, so every client must be able to
\begin{inparaenum}
\item create their local database,
\item present the user with forms for data entry,
\item request missing information from the server, and
\item add information to the server.
\end{inparaenum}
Since the clients will be communicating with the API over HTTP, the server-side
methods will be accessed through URLs of the form
\mbox{\texttt{http://domain.com/api/name}}, where \texttt{name} will be
the name of a particular resource. These client needs translate into the
following server side methods.

%--------------------------------------------------------
\subsubsection{/api/schema}  \label{sec:api_schema}
This resource will return the information that the clients need in order to
build their local databases and replicate the server's database.

%--------------------------------------------------------
\subsubsection{/api/metamodel}  \label{sec:api_metamodel}
This resource will return the information that clients need in order to present
and gather information to and from users.

%--------------------------------------------------------
\subsubsection{/api/sync}  \label{sec:api_sync}
This resource will provide clients with all of the database rows that are
currently absent from their local databases.

%--------------------------------------------------------
\subsubsection{/api/create}  \label{sec:api_create}
Clients will send requests to create rows on the central server to this
resource.

Figure~\ref{fig:client_states} is a state based diagram that shows how clients can
learn how to create their local databases, display information to users,
synchronize themselves and create new information on the server. 

\begin{figure}[h!]
\centering
\begin{tikzpicture}[node distance = 3cm, semithick, auto, scale=.7, every node/.style={transform shape}]
  % Place nodes
  \node [dot] (start) {};
  \node [block, right of=start, node distance=2cm] (s1) {schema};
  \node [block, right of=s1] (s2) {metamodel};
  \node [block, right of=s2] (s3) {sync};
  \node [block, right of=s3] (s4) {create};
 
  \path [line] (start) -- node {1} (s1) ;
  \path [line] (s1) -- node {2} (s2);
  \path [line] (s2) -- node {3} (s3);

  \path (s3) edge [loop above] node {4} (s3);
  \path (s3) edge [line,bend left] node {5} (s1);

  \path (s3) edge [line, bend left] node {6} (s4);
  \path (s4) edge [line, bend left] node {7} (s3);

\end{tikzpicture}
\caption{A state based representation of Coexist clients. Using only the four
API methods provided, a client can build its local database, learn how to
display it, sync itself and create new information on the server.}
\label{fig:client_states}
\end{figure}

Clients start out on edge 1 without any information. By the time they reach edge
3, they have everything they need in order to build local databases for
replicating the server, as well as metadata relating to the display of the
database contents. The next step is a call to \texttt{/api/sync} to pull down
missing rows from the server (which will be all rows on the first run).
Consecutive calls allow clients to stay in sync while other clients make
changes to the database. Every time a row is created on the database through
a call to \texttt{/api/create}, it resynchronizes itself and imports the newly
created rows into its local database. At this point it can take edge 5 to
restart the entire process, allowing it to respond to changes in the server's
database schema.

So far, we have a rough idea of what clients need and what the server
must provide them, but not enough to know exactly how to implement these methods
on the server. For example, we have questions about
\begin{inparaenum}
\item\label{enum:data} what the data exchange format between clients and the server will be, 
\item\label{enum:sync} how to synchronize data, and
\item\label{enum:meta} how to tell clients to display the synced data to users. 
\end{inparaenum}

The next section will answer questions \ref{enum:data} and
\ref{enum:sync} by introducing a simple method of communication that allows
clients to maintain the integrity of their local databases, as well as a
stateless synchronization protocol for relational databases that can be
implemented in pure SQL. In Section~\ref{sec:metamodel}, we answer
question~\ref{enum:meta} by defining a JSON based metamodel that contains all of
the information necessary for clients to display and collect data from users.
Finally, in Section~\ref{sec:deployment}, the build process will be described.




