%--------------------------------------------------------
\section{The Server and API} \label{sec:api}
%--------------------------------------------------------

In the Coexist client-server model, the clients communicate with the server
through an HTTP based API. This section will specify the details of the
communication between the server and clients, starting with the declaration of
necessary API methods in Section~\ref{sec:api_methods}.


%--------------------------------------------------------
\subsection{API Methods}  \label{sec:api_methods}

% declaring the api methods, not implementing


% maybe introduce the data replication stuff before this point. 
The API can be designed by taking the client's needs into
consideration. In Coexist, every client has a local copy of their central
database, so every client must be able to
\begin{inparaenum}
\item create their local database,
\item present the user with forms for data entry,
\item request missing information from the server, and
\item add information to the server.
\end{inparaenum}
Since the clients will be communicating with the API over HTTP, the server-side
methods will be accessed through URLs of the form
\mbox{\texttt{http://domain.com/api/name}}, where \texttt{name} will be
the name of a particular resource. These client needs translate roughly into the
following server side methods.

%--------------------------------------------------------
\subsubsection{/api/schema}  \label{sec:api_schema}
This resource will return the information that the clients need in order to
build their local databases and replicate the server's database.

%--------------------------------------------------------
\subsubsection{/api/metamodel}  \label{sec:api_metamodel}
This resource will return the information that clients need in order to present
and gather information to and from users.

%--------------------------------------------------------
\subsubsection{/api/sync}  \label{sec:api_sync}
This resource will provide clients with all of the database rows that are
currently absent from their local databases.

%--------------------------------------------------------
\subsubsection{/api/create}  \label{sec:api_create}
Clients will send requests to create rows on the central server to this
resource.

Figure~\ref{fig:client_states} is a state based diagram that shows how clients can
learn how to create their local databases, display information to users,
synchronize themselves and create new information on the server. 

\begin{figure}[h!]
\centering
\begin{tikzpicture}[node distance = 3cm, semithick, auto, scale=.7, every node/.style={transform shape}]
  % Place nodes

  \node [dot] (start) {};
  \node [block, right of=start, node distance=2cm] (s1) {schema};
  \node [block, right of=s1] (s2) {metamodel};
  \node [block, right of=s2] (s3) {sync};
  \node [block, right of=s3] (s4) {create};
 
  \path [line] (start) -- node {1} (s1) ;
  \path [line] (s1) -- node {2} (s2);
  \path [line] (s2) -- node {3} (s3);

  \path (s3) edge [loop above] node {4} (s3);
  \path (s3) edge [line,bend left] node {5} (s1);

  \path (s3) edge [line, bend left] node {6} (s4);
  \path (s4) edge [line, bend left] node {7} (s3);

\end{tikzpicture}
\caption{A state based representation of Coexist clients. Using only the four
API methods provided, a client can build its local database, learn how to
display it, sync itself and create new information on the server.}
\label{fig:client_states}
\end{figure}

Clients start out on edge 1 without any information. By the time they reach edge
3, they have everything they need in order to build local databases for
replicating the server, as well as metadata relating to the display of the
database contents. The next step is a call to \texttt{/api/sync} to pull down
missing rows from the server (which will be all rows on the first run).
Consecutive calls allow clients to stay in sync while other clients make
changes to the database. Every time a row is created on the database through
a call to \texttt{/api/create}, it resynchronizes itself and imports the newly
created rows into its local database. At this point it can take edge 5 to
restart the entire process, allowing it to respond to changes in the server's
database schema.

So far, we have a rough idea of what clients need and what the server
must provide them, but not enough to know exactly how to implement these methods
on the server. For example, we have questions about
\begin{inparaenum}
\item\label{enum:data} what the data exchange format between clients and the server will be, 
\item\label{enum:sync} how to synchronize data, and
\item\label{enum:meta} how to tell clients to display the synced data to users. 
\end{inparaenum}

The next section will answer questions \ref{enum:data} and
\ref{enum:sync} by introducing a simple method of communication that allows
clients to maintain the integrity of their local databases, as well as a
stateless synchronization protocol for relational databases that can be
implemented in pure SQL. In Section~\ref{sec:metamodel}, we answer
question~\ref{enum:meta} by defining a JSON based metamodel that contains all of
the information necessary for clients to display and collect data from users.
Finally, in Section~\ref{sec:deployment}, the build process will be described.
%Finally, in Section~\ref{sec:api_imp}, these four API methods are revisited and
%fully implemented.



%
%%-------------------------------------------------------
%\subsection{Client-Server Communication}
%
%%Many mobile applications are supplemented by a local database (typically
%%SQLite). Some applications are stand alone and use these databases primarily as
%%persistent storage; the Networkmaine inventory application uses its local
%%database to interface with the remote back end MySQL server that it depends
%%heavily on. There were three methods for this interfacing that represent
%%different degrees of caching, from none, to selective, to complete.
%
%
%%Before deciding upon a
%
%This section will explore some of the methods that the client can use to
%interface with the server, their advantages and disadvantages, and explain why
%the method labeled as the \textit{mirror method} is the appropriate choice. The
%following subsections will give details on the implementation of this method and
%its implications on the database schema; particularly, how normalized it should
%be and how to handle deletions. In all diagrams in this section, the arrows
%indicate the flow of data and the labels indicate the actions that cause them.
%
%
%%-------------------------------------------------------
%\subsubsection{Window method}
%
%In the window method, the data that a user sees at any given time is a window
%into the back end server; all of the data came from the server and none of it is
%going to persist beyond this screen. The major advantage to this method is its
%ease of implementation. This would be a suitable fit for a small weather
%application; in fact, it is the approach that my own weather application ,
%QtWeather, uses.  
%
%%TODO add some more analysis details
%
%\begin{figure}[h!]
%\centering
%\begin{tikzpicture}[node distance = 2cm, auto, scale=.7, every node/.style={transform shape}]
%  % Place nodes
%  \node [block] (client) {client};
%  \node [block, right of=client, node distance=5cm] (server) {server};
%  % Draw edges
%  \path [line] (server.165) -- node [above] {query} (client.15);
%  \path [line] (client.345) -- node [below] {udpate} (server.195);
%\end{tikzpicture}
%\caption{Window method}
%\label{fig:window}
%\end{figure}
%
%
%%-------------------------------------------------------
%\subsubsection{Cache method}
%
%Treating the client as a window requires that we don't have large amounts of
%data constantly being retrieved from the server. If this is not the case, then
%we may want to identify some data (or queries) that are commonly used in order
%to cache (and optionally sync) on the client's local database. This would be
%suitable for a mobile social networking client. Take a typical Facebook
%application for example; it would be advantageous to cache a users top five
%commonly used groups as well as their friend list because this information is
%supposedly accessed often and is relatively static. In contrast, caching the
%Facebook news feed would probably result in a lot of wasted space because of how
%fast it changes and how rarely old news feed entries are viewed. 
%
%
%
%\begin{figure}[h!]
%\centering
%\begin{tikzpicture}[node distance = 2cm, auto, scale=.7, every node/.style={transform shape}]
%  % Place nodes
%  \node [block] (client) {client};
%  \node [block, right of=client, node distance=5cm] (server) {server};
%  \node [db, below of=client, node distance=4cm, align=center] (database) {local \\ database};
%
%  % Draw edges
%  % Need to align in order to line break
%  \path [line] (server.165) -- node [above, align=center] {query \\ un-cachable data} (client.15);
%  \path [line] (client.345) -- node [below] {update} (server.195);
%  \path [line] (server) |- node {sync cache} (database);
%  \path [line] (database) -- node [align=left] {query frequent \\ and static data} (client);
%\end{tikzpicture}
%\caption{Cache method}
%\label{fig:cache}
%\end{figure}
%
%
%%-------------------------------------------------------
%\subsubsection{Mirror method}
%The window method has the client retrieve all data from the server all of the
%time, the cache method identifies some useful data to store and then caches it
%locally; the mirror method is the next natural step in this progression: cache
%the entire database. There are some immediate concerns with this approach, most
%prominent of which concerns the size of the database potentially being much
%larger than the what we can reasonably expect to store on the client. The
%advantages to this method are faster query times and less reliance on the
%network. In fact, this is the only method that would enable a fully readable
%database off line.
%
%\begin{figure}[h!]
%\centering
%\begin{tikzpicture}[node distance = 2cm, auto, scale=.7, every node/.style={transform shape}]
%  % Place nodes
%  \node [block] (client) {client};
%  \node [block, right of=client, node distance=5cm] (server) {server};
%  \node [db, below of=client, node distance=4cm, align=center] (database) {local \\ database};
%
%  % Draw edges
%  % Need to align in order to line break
%  \path [line] (client) -- node {udpate} (server);
%  \path [line] (server) |- node {mirror} (database);
%  \path [line] (database) -- node [align=left] {query} (client);
%\end{tikzpicture}
%\caption{Mirror method}
%\label{fig:mirror}
%\end{figure}
%
%The Networkmaine inventory database contains information for approximately six
%thousand devices, location, purchase order and rma information for batches of
%devices; the size of the MySQL database dump is 1.5 megabytes. In addition, the
%database will not grow quickly. The size of the database makes the mirror method
%very attractive: if Networkmaine added two thousand units a year to its
%deployment then in twenty years the size of the database would be no less than
%12 megabytes; still reasonable to store on the mobile clients we see available
%today (probably more so on the mobile clients in twenty years, if we still use
%them). Realistically, there won't be more than a few hundred devices added to
%the inventory each year, at most. It is for these reasons that the mirror method
%is preferable for this project. 
%
%
%
%
%%-------------------------------------------------------
%\subsection{Implementation}
%
%Many database systems support mirroring, or data replication. The driving factor
%in this technology seems to be data availability: always having a slave database
%server that can assume the primary role in the event that something goes
%wrong\cite{microsoft_database_????}. The issue here is that we don't have much
%control over the development environment; the server side database is MySQL
%because that is what the staff is familiar with and the client side database is
%SQLite because that is what Android supports. While MySQL does support data
%replication to MySQL slave servers, it certainly does not support replication to
%arbitrary alternative DBMS\cite{mySQL_mySQL_????}.
%
%%That does not mean it is impossible. We considered an interesting solution from
%%fyreCloud's Amsler project. It was designed with Android in mind, attempting to
%%mediate the MySQL replication protocol to make it compatible with
%%SQLite\cite{fyrecloud_solutions_fyrecloud_????}. The actions of the Amsler
%%project can be summarized as follows.
%%
%%\begin{enumerate}
%%\item Configure the MySQL back end for data replication as described by the MySQL manual. 
%%\item Use TCP/IP connections to continuously push changes to the client in the form of MySQL commands.
%%\item Use a lexer and parser to map the incoming MySQL queries to valid SQLite queries. 
%%\end{enumerate}
%%
%%Step 3 was troublesome; the syntax of different SQL implementations has many
%%subtle differences. The idea of having a limited set of queries (dictated by the
%%Amsler project's progress) that we could use on the server was too much to bear.
%
%A simpler approach to this problem involved the addition of \textit{metacolumns}
%to the original schema; or columns defined in a SQL table that do not
%traditionally relate to the other columns in that table, but instead are used
%for some application level purpose. The most important of these metacolumns is
%the \texttt{mod\_ts} (modification timestamp) column. Let's take a simple
%relation \hbox{\texttt{Student(id, name, year)}} and examine how it can be
%synchronized across multiple instances without using any DBMS specific features.
%In this example, both the client and server tables are initialized to the table
%in Figure~\ref{fig:student_init}. 
%
%
%\begin{figure}[h!]
%\center
%\begin{tabular}{ l  l  l  l }
%id  & name      & year  & \textit{mod\_ts} \\
%\hline
%1   & Anthony   & 4     & \textit{0}        \\
%2   & Lucas     & 3     & \textit{0}        \\
%3   & Steven    & 3     & \textit{0}        \\
%\end{tabular}
%\caption{Initial state of the \texttt{Student} table.}
%\label{fig:student_init}
%\end{figure}
%
%As implied, the \texttt{mod\_ts} metacolumn will be updated with the time of the
%transaction. For the purpose of readability, we represent this timestamp as the
%number of seconds since the creation of the table. 
%
%Now lets update the server from one of the many possible client applications. 
%
%
%\begin{figure}[h!]
%\begin{lstlisting}[language=sql]
%BEGIN TRANSACTION;
%INSERT INTO Student(name,year) VALUES(Ryan, 4);
%UPDATE Student SET year=4 WHERE name='Lucas';
%COMMIT;
%\end{lstlisting}
%\caption{Transaction on the server.}
%\label{fig:student_transaction1}
%\end{figure}
%
%It is worth mentioning now that the following trigger must exist on the server
%database for \texttt{INSERT}, \texttt{UPDATE} and \texttt{REPLACE} (but not
%necessarily for \texttt{DELETE}, explained in section \ref{sec:schema}).
%
%
%\begin{figure}[h!]
%\begin{lstlisting}[language=sql]
%CREATE TRIGGER mod_ts AFTER INSERT ON Student
%  FOR EACH ROW SET mod_ts = NOW();
%\end{lstlisting}
%\caption{Trigger for modification timestamps.}
%\label{fig:student_trigger}
%\end{figure}
%
%The transaction of figure~\ref{fig:student_transaction1} will leave the server
%in the following state.
%
%\begin{figure}[h!]
%\center
%\begin{tabular}{ l  l  l  l }
%id  & name      & year  & \textit{mod\_ts} \\
%\hline
%1   & Anthony   & 4     & \textit{0}        \\
%2   & Lucas     & 4     & \textit{10}        \\
%3   & Steven    & 3     & \textit{0}        \\
%4   & Ryan      & 4     & \textit{10}        \\
%\end{tabular}
%\caption{Server state after transaction of figure \ref{fig:student_transaction1}.}
%\label{fig:student_update}
%\end{figure}
%
%Currently all clients are still in the state of figure~\ref{fig:student_update}
%and need to be synchronized with the server; this is a multi-part process,
%central to which is the client side query in figure~\ref{fig:student_change}. 
%
%\begin{figure}[h!]
%\begin{lstlisting}
%SELECT max(mod_ts) FROM Student;
%\end{lstlisting}
%\caption{Find the most recent local change.}
%\label{fig:student_change}
%\end{figure}
%
%In our case (and in many cases), the client communicates with the server
%database through a RESTful API. One such function provided by this API is
%$getChanges(table,~ts)$, where $table$ is the name of a table and $ts$ is a
%timestamp. In our running example, all clients would have to send a call to
%$getChanges(~Student,~0)$, where $0$ was returned from the query of
%figure~\ref{fig:student_change}. The server will respond to the client by
%returning all rows of the Student table where $mod\_ts > 0$, enabling the client
%to insert or replace them locally. If the most recent change on the client side
%is the same as the most recent change on the server then they are in
%sync\footnote{Recall from figure~\ref{fig:mirror} that the client never modifies
%its local database, it only requests synchronization from the server.}.
%
%
%
%%--------------------------------------------------------
%\subsection{Handling Deletes}  \label{sec:}
%
%Section~\ref{sec:api} introduced the Mirror Method and discussed an
%implementation that would allow stateless synchronization with multiple clients
%using pure SQL. An issue arose when trying to incorporate SQL \texttt{DELETE}s into the
%system: because server side changes were synchronized with clients using
%\mbox{modification} timestamps, deleted rows would not be propagated to clients. In
%order to support the DELETE operation and preserve the synchronization, an
%additional metacolumn was added: \textit{deleted}.  This metacolumn tells the
%Coexist clients to hide rows from the interface, giving the illusion of
%deletion. This modification would turn Figure~\ref{fig:student_update} into
%Figure~\ref{fig:new_deleted}.
%
%\begin{figure}[h!]
%\center
%\begin{tabular}{ l  l  l  l  l}
%id  & name      & year  & \textit{mod\_ts} & \textit{deleted} \\ 
%\hline
%1   & Anthony   & 4     & \textit{0}   & \textit{0}     \\
%2   & Lucas     & 4     & \textit{10}   & \textit{0}     \\
%3   & Steven    & 3     & \textit{0}   & \textit{1}     \\
%4   & Ryan      & 4     & \textit{10}   & \textit{0}     \\
%\end{tabular}
%\caption{Introducing the \textit{deleted} metacolumn.}
%\label{fig:new_deleted}
%\end{figure}
%
%In this example, \texttt{Steven} would have been marked as deleted from the
%database and hidden from the user interface. Because the change happens in a
%column, it can be propagated out to the server and ultimately the other clients
%without having to change the modification-timestamp-based approach decided in
%Section~\ref{sec:api}.
%
%Unfortunately, two additional problems arose from this fix: 
%\begin{inparaenum}
%\item The deleted rows build up on the server.
%\item The primary keys are still occupied. 
%\end{inparaenum}
%The buildup of rows can be remedied by periodically deleting all rows marked as
%deleted and triggering resynchronizations on the clients. This would cause them
%to delete their local databases and download all missing rows (which would no
%longer include deleted rows). This is only necessary if the deleted rows start
%to occupy a significant amount of space.
%
%The primary key issue was more serious. In Figure~\ref{fig:new_deleted}, clients
%would no longer be able to add a \texttt{Steven} to the database after marking
%him as deleted, nor could they modify the row to un-delete him (assuming
%\texttt{name} is a primary key). The solution involved a small expansion of the
%API specification: the addition of a \texttt{replace} parameter to all requests
%made to \texttt{/api/create/}\footnote{Recall that this is the API method used
%for creating rows on the server.} from clients. Figure~\ref{fig:create} depicts
%  the actions of the API when receiving requests to \texttt{/api/create/}.
%
%
%\begin{figure}[h!]
%\centering
%\begin{tikzpicture}[node distance = 3cm, auto, scale=.7, every node/.style={transform shape} ]
%  % Place nodes
%  \node [block] (request) {Request to /api/create/};
%  \node [decision, below of=request] (isconflict) {Key violation?};
%  
%  \node [block, left of=isconflict] (create) {Create the row.};
%  \node [decision, right of=isconflict] (nocreate) {Does replace=1?};
% 
%  \node [block, below of=create] (ok) {\textbf{200 OK}}; 
%
%  \node [dot, below of=nocreate] (nocreateDOT) {};
%
%  \node [block, left of=nocreateDOT] (replace) {Replace it};
%  \node [decision, below of=replace] (isreplace) {Does deleted=1};
%
%  \node [block, left of=isreplace] (error) {\textbf{409 Conflict}};
%
%  % Draw edges
%  \path [line] (request) -- (isconflict);
%  \path [line] (isconflict) -- node [above] {no} (create);
%  \path [line] (isconflict) -- node [above] {yes} (nocreate);
%  \path [line] (create) -- (ok);
%
%  \path [edge] (nocreate) -- (nocreateDOT);
%  \path [line] (nocreateDOT) -- node [above] {yes} (replace);
%  \path [line] (replace) -- (ok);
%
%  \path [line] (nocreateDOT) |- node [right] {no} (isreplace);
%  \path [line] (isreplace) -- node [right] {yes} (replace);
%
%  \path [line] (isreplace) -- node [above] {no} (error);
%
%\end{tikzpicture}
%\caption{API process for row creation.}
%\label{fig:create}
%\end{figure}
%
%The flow chart starts with ``Request to /api/create/'', which is a request
%to create a row on the server from a client. If there is no key constraint
%violation, then the row is created and the server responds with a \texttt{200
%OK}. If there is a violation then the server checks to see if the request's
%\texttt{replace} parameter is set to \texttt{1}. If it is then the conflicting
%row can be replaced without worry and the server responds with a \texttt{200
%OK}.  If not then the server checks to see if the conflicting row is marked as
%deleted. If it is, then the conflicting row can be replaced and the server
%responds with a \texttt{200 OK}. If not, then the server responds with a
%\texttt{409 CONFLICT} and the row is not created. At this point, the client has
%enough information to ask the user if they wish to replace (or update) the
%conflicting row on the server and potentially resend the request with
%\texttt{replace=1}. 
% 
%This solution reduces the functional burden of the deleted rows, enables SQL
%\texttt{UPDATE} functionality, is completely representable with pure SQL, and
%requires no significant changes to the existing API. 
%
%
%
%
%%--------------------------------------------------------
%\subsection{Database Replication} \label{sec:db}
%
%
%It is important to ensure that the client can never update the server if the
%client is out of sync, and the client must always have a way of re-syncing
%itself.  This can be achieved with two simple API functions, \texttt{/api/sync}
%and \texttt{/api/schema}; where one function requests missing information and
%another requests schema information. This design accounts for the database
%schema being upgraded on the server, and clients attempting to make changes to
%the database server when out of sync. Figure~\ref{fig:protocol} depicts the
%network dialog between a client and server where the client attempts to sync
%itself when it is using and old version of the database. 
%
%
%
%
%\begin{figure}[h!]
%\centering
%\begin{tikzpicture}[node distance = 2cm, auto, scale=.7, every node/.style={transform shape}]
%  % Place nodes
%  \node [block] (client) {client};
%  \node [block, right of=client, node distance=6cm] (server) {server};
%  
%  \node [dot, below of=client, node distance=2.5cm, label=180:1] (c1) {};
%  \node [dot, below of=server, node distance=2.5cm] (s1) {};
%  \path [line] (c1) -- node [above, align=left] {
%    \tt\textbf{GET /api/sync} \\ 
%    \tt version:1 \\ 
%    \tt tables:[t1,t2] \\ 
%    \tt mod\_ts:[33,12]}   (s1);
%  \path [edge] (client) -- (c1);
%  \path [edge] (server) -- (s1);
%
%  \node [dot, below of=c1, node distance=1cm, label=180:2] (c2) {};
%  \node [dot, below of=s1, node distance=1cm] (s2) {};
%  \path [line] (s2) -- node [above] {\tt\textbf{409 CONFLICT}} (c2);
%  \path [edge] (c2) -- (c1);
%  \path [edge] (s2) -- (s1);
%
%  \node [dot, below of=c2, node distance=1.5cm, label=180:3] (c3) {};
%  \node [dot, below of=s2, node distance=1.5cm] (s3) {};
%  \path [line] (c3) -- node [above, align=left] {
%    \tt\textbf{GET /api/schema} \\ 
%    \tt db:SQLite            } (s3);
%  \path [edge] (c2) -- (c3);
%  \path [edge] (s2) -- (s3);
%
%  \node [dot, below of=c3, node distance=2cm, label=180:4] (c4) {};
%  \node [dot, below of=s3, node distance=2cm] (s4) {};
%  \path [line] (s4) -- node [above, align=left] {
%    \tt\textbf{200 OK} \\
%    \tt\{version:2 \\
%    \tt~SQL:[CREATE TABLE...]\} } (c4);
%  \path [edge] (c3) -- (c4);
%  \path [edge] (s3) -- (s4);
%    
%  \node [dot, below of=c4, node distance=2cm, label=180:5] (c5) {};
%  \node [dot, below of=s4, node distance=2cm] (s5) {};
%  \path [line] (c5) -- node [above, align=left] {
%    \tt\textbf{GET /api/sync} \\ 
%    \tt version:2 \\ 
%    \tt tables:[t1,t2,t3] \\ 
%    \tt mod\_ts:[0,0,0]}   (s5);
%
%  \path [edge] (c5) -- (c4);
%  \path [edge] (s4) -- (s5);
%
%
%  \node [dot, below of=c5, node distance=2cm, label=180:6] (c6) {};
%  \node [dot, below of=s5, node distance=2cm] (s6) {};
%  \path [line] (s6) -- node [above, align=left] {
%    \tt\textbf{200 OK} \\
%    \tt \{t1:[(..),(..)] \\
%    \tt~t2:[] \\
%    \tt~t3:[] \}} (c6);
%  \path [edge] (c6) -- (c5);
%  \path [edge] (s6) -- (s5);
%
%\end{tikzpicture}
%\caption{Synchronization protocol}
%\label{fig:protocol}
%\end{figure}
%
%The actions can be described as follows. 
%
%\begin{enumerate}
%
%\item The client sends a GET request to /api/sync on the server, including the database version it is using, an array of table names and an array of maximum modification timestamps for those tables (found in a metacolumn).
%
%\item The server responds with a 409 error message because it is currently using version 2 of the database, while the client is using version 1.
%
%\item The client sends a GET request to /api/schema on the server to get the current schema that the server side database is using. It requests the SQL in SQLite syntax.
%
%\item The server responds with a 200 OK code. The body of the response is a JSON object containing the version of the server's database and the SQL statements that need to be executed in SQLite to build it. The constraints do not need to be sent to the client, only the table declarations (and possibly views). Since the client never updates its own database through SQL, the constraints are not necessary. 
%
%\item The client now has enough information to re-create its database. Since there is nothing on the client that isn't also on the server, the client can simply replace the contents of its database with this new schema. A call to /api/sync is repeated with the new schema. 
%
%\item There is no version conflict; the server responds with a 200 OK code. The body of the response is a JSON object that maps table names to arrays containing all of the tuples that the client is missing. 
%
%\end{enumerate}
%
%
%The dialog would look slightly different for an out of sync client attempting to
%make server changes. In figure~\ref{fig:protocol}, the client supplies the
%server with all of the tables and their most recent update timestamps. If the
%client was to send this information with each API request then the server could
%similarly notify the client when they appear out of date; but timestamps and
%table names can be verbose, it would better to send a representative checksum in
%their place. Figure~\ref{fig:signature} represents the sha1 checksum that can be
%sent in API requests as a \textit{signature} to validate the client; $T$
%represents the concatenation of all table names, and $M$ represents the
%concatenation of all maximum modification timestamps. The first timestamp should
%correspond to the first table, and so on.
%
%\begin{figure}[h!]
%\[
%sha1(version + T + M + username + sha1(password))
%\]
%\caption{API request signature.}
%\label{fig:signature}
%\end{figure}
%
%So for a schema that consisted of the tables $Students$ and $Courses$, with
%maximum modification timestamps\footnote{These timestamps are the same ones
%depicted in figure~\ref{fig:student_init}} of $2012-12-01 16:18:31$ and
%$2012-12-01 16:18:43$ respectively, under version $1$ of the database, with a
%user name of $anthonyn$ and a password of $password$, the signature would be
%equal to \hbox{$9d64e5fbba865b491adeeeead91666df8d7cb1ec$}.
%
%%begin{figure}[h!]
%%sha1("1StudentsCourses2012-12-01 16:\\
%%8:312012-12-01 16:18:43anthonyn5baa6\\
%%e4c9b93f3f0682250b6cf8331b7ee68fd8") \\
%% 9d64e5fbba865b491adeeeead91666df8d7cb1ec$
%%caption{My Listing}
%%label{fig:list}
%%end{figure}
%
%
%
%%The popular client authentication method today is the OAuth
%%library\cite{_oauth_????} used by Facebook, Twitter and LinkedIn to name a few.
%%The idea is to use authentication tokens in place of passwords so that third
%%party applications can access someone's information by making requests with
%%tokens instead of passwords. For the purposes of Networkmaine, there will be no
%%third party plug in service and all users access the same information over SSL.
%%Storing local password hashes or requesting passwords per transaction (depending
%%on user preferences) will suffice.
%
%
%
%
%
%
%
%%-------------------------------------------------------
%\subsection{Schema Implications}
%\label{sec:schema}
%
%% users won't have to create schemas.
%
%The synchronization depends on a few assumptions. If a row was deleted from the
%server then there would be no reliable way for the client to discover this using
%only the metacolumns. Assuming no insertions have taken place, missing rows
%would be $C - S$, where $C$ is the client side table and $S$ is the server side,
%but we would like to avoid sending the contents of entire tables over the
%network (aside from the initial synchronization). 
%
%Many of the relations of our schema include timestamps to enable us to pose more
%interesting queries. For example, our \texttt{Shelved( serial\_no, date)}
%relation allows us to tell Networkmaine all of the devices that were shelved in
%a given time range; in the past, these devices might have been deleted from
%their records entirely. For this reason, and the reason in the prior paragraph,
%we decided it would be a good idea to never delete any data from the database.
%Instead, the \texttt{deleted} metacolumn was added to each relation; a simple
%boolean toggle that tells the application that this tuple should be considered
%deleted. This enables our synchronization method and preserves the integrity of
%queries that rely on having sufficient data to operate on.
%
%Another practice to avoid is using relations to store states. In the original
%schema for Networkmaine, there was a table \hbox{\texttt{Shelved (serial, date,
%note)}} that was used to track devices that should be considered retired. These
%devices could potentially be used again and would have to be marked as deleted
%in the \texttt{Shelved} table with metacolumns. It is also reasonable to assume
%that the \texttt{Shelved} table's growth will mirror the \texttt{Device} table's
%growth because each new device will potentially replace an older device. 
%
%One possible solution would be to merge the \texttt{Shelved} table into the
%\texttt{Device} table. In MySQL, a \texttt{BOOLEAN} occupies 1 byte, a
%\texttt{TIMESTAMP} occupies 4 bytes, and \texttt{VARCHAR} occupies at least 0
%bytes and at most 3 bytes per character in utf-8 under the InnoDB engine. The
%initial database will have about six thousand devices in it, so merging the
%tables would add about $6000~bytes + 24000~bytes + 625~bytes \approx 30~kb$, or
%5 bytes per row, where the $625~bytes$ came from the sum off all of the 1 bit
%bitmasks added for allowing \texttt{NULL} values in the \texttt{VARCHAR} column;
%a negligible amount for the small growth rate of the database. The updated table
%is shown in figure~\ref{fig:updated_device}.
%
% 
%\begin{figure}[h!]
%\begin{lstlisting}[language=sql]
%CREATE TABLE Device(
%  serial VARCHAR(32) PRIMARY KEY,
%  tag VARCHAR(32) NOT NULL,
%  description VARCHAR(32),
%  shelved BOOL DEFAULT FALSE NOT NULL,
%  shelved_notes VARCHAR(32),
%  shelved_date TIMESTAMP);
%\end{lstlisting}
%\caption{Updated \texttt{Device} table.}
%\label{fig:updated_device}
%\end{figure}
%
%
