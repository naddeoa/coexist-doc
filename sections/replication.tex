%--------------------------------------------------------
\section{Database Replication} \label{sec:replication}
%--------------------------------------------------------


%Many database systems support mirroring, or data replication. The driving factor
%in this technology seems to be data availability: always having a slave database
%server that can assume the primary role in the event that something goes
%wrong\cite{microsoft_database_????}. The issue here is that we don't have much
%control over the development environment; the server side database is MySQL
%because that is what the staff is familiar with and the client side database is
%SQLite because that is what Android supports. While MySQL does support data
%replication to MySQL slave servers, it certainly does not support replication to
%arbitrary alternative DBMS\cite{mySQL_mySQL_????}.


%A simpler approach to this problem involved the addition of \textit{metacolumns}
%to the original schema; or columns defined in a SQL table that do not
%traditionally relate to the other columns in that table, but instead are used
%for some application level purpose. The most important of these metacolumns is
%the \texttt{mod\_ts} (modification timestamp) column. Let's take a simple
%relation \hbox{\texttt{Student(id, name, year)}} and examine how it can be
%synchronized across multiple instances without using any DBMS specific features.
%In this example, both the client and server tables are initialized to the table
%in Figure~\ref{fig:student_init}. 

The implementation of \schema reduces to the problem of database replication.
As mentioned in Section~\ref{sec:related}, various DBMSs have replication
functionality, and companies have attempted to mediate the protocols that they
use in order to enable replication across different DBMSs with varying success.
A commitment to the built in database replication features means a commitment to
a single DBMS per instance of Coexist and support limited to those DBMSs that
have replication features to begin with, which would complicate the likely
scenario of Android SQLite clients interfacing with a MySQL server database. 
Instead, a stateless, pure SQL solution is preferred. The remainder of
this section will describe such an solution and its role in implementing
the \schema method.


%--------------------------------------------------------
\subsection{Representing Changes in Rows}  \label{sec:changes}

At the core of a stateless SQL replication protocol lies the ability to import
database changes using only SQL statements. To simplify the problem, we assume that
\begin{inparaenum}
\item each row stores a timestamp value of its most recent modification and
\item rows are not being deleted, only created and updated.
\end{inparaenum}
For every table in a database there will be an additional column called
\modts of type \term{timestamp}. We will refer to columns that provide
meta-information about their row as \term{metacolumns}.
Figure~\ref{fig:student_init} is a SQL table that represents students in a
university. For readability, the \modts column contains the number of seconds
since the creation of the table, instead of a lengthy timestamp.

\begin{figure}[h!]
\center
\begin{tabular}{ l  l  l  l }
id  & name      & year  & \textit{mod\_ts} \\
\hline
1   & Anthony   & 4     & \textit{0}        \\
2   & Lucas     & 3     & \textit{0}        \\
3   & Steven    & 3     & \textit{0}        \\
\end{tabular}
\caption{A simple table of that represents students in a university. The \modts
column is a metacolumn that represents the most recent modification time.}
\label{fig:student_init}
\end{figure}

For some DBMS, a trigger similar to the one in Figure~\ref{fig:student_trigger}
may be required for \texttt{INSERT}, \texttt{UPDATE}, and \texttt{REPLACE} while
others, like MySQL, happen to automatically update select timestamp columns by
default\cite{_mysql_????-2}.

\begin{figure}[h!]
\begin{lstlisting}[language=sql]
CREATE TRIGGER mod_ts AFTER INSERT ON Student
  FOR EACH ROW SET mod_ts = NOW();
\end{lstlisting}
\caption{A trigger for updating the value of a \modts column with the current
timestamp whenever an insert is performed.}
\label{fig:student_trigger}
\end{figure}

With this metacolumn in place, the only piece of information the server needs about the
client is the value of its highest modification timestamp. Then, it can retrieve
the missing rows for the client using a simple SQL query: \texttt{SELECT * FROM
Students WHERE mod\_ts > x}, assuming the name of the table in
Figure~\ref{fig:student_init} is \term{Students}, and $x$ is the maximum
modification timestamp sent from the client. To see how this works, consider
Figure~\ref{fig:student_update} where a SQL statement is used to update row 2
and add for 4, ten seconds after the creation of the database. The resulting
database is shown in tabular format below the SQL.

\begin{figure}[h!]
\begin{lstlisting}[language=sql]
BEGIN TRANSACTION;
INSERT INTO Students(name,year) VALUES(Ryan, 4);
UPDATE Students SET year=4 WHERE name='Lucas';
COMMIT;
\end{lstlisting}
\center
\begin{tabular}{ l  l  l  l }
id  & name      & year  & \textit{mod\_ts} \\
\hline
1   & Anthony   & 4     & \textit{0}        \\
2   & Lucas     & 4     & \textit{10}        \\
3   & Steven    & 3     & \textit{0}        \\
4   & Ryan      & 4     & \textit{10}        \\
\end{tabular}
\caption{The state of the \term{Students} table after an insert and update are
performed. The \modts values of rows 2 and 4 are updated accordingly.}
\label{fig:student_update}
\end{figure}

A client that is still in the state of Figure~\ref{fig:student_init} and
requesting a sync would supply the server with a maximum modification
timestamp for its \term{Students} table (a value of 0) and the server would reply with
rows 2 and 4. After each sync, the client's database should exactly reflect that
of the servers, so there is no need to respect primary key constraints -- all
client rows can be replaced upon conflict. The next step is to introduce SQL
deletes into the example and modify the algorithm to operate correctly.



%--------------------------------------------------------
\subsection{Handling Deletes}  \label{sec:}


An issue arose when trying to incorporate SQL \texttt{DELETE}s into the
system: because server side changes were synchronized with clients using
\mbox{modification} timestamps, deleted rows would not be propagated to clients. In
order to support the DELETE operation and preserve the synchronization, an
additional metacolumn was added: \textit{deleted}.  This metacolumn tells the
Coexist clients to hide rows from the interface, giving the illusion of
deletion. 

If any of the columns in Figure~\ref{fig:student_update} are deleted, the server
would not be able to propagate changes to clients because there would be no
modification timestamp change. Instead, a second metacolumn, called
\term{deleted} can be introduced to represent deletions. Specifically, deletes
are now accomplished through updates of the \term{deleted} column, instead of
SQL \var{DELETE} queries, and the Coexist clients can just as easily hide deleted
columns from the user interface. Figure~\ref{fig:new_deleted} represents the
\term{Students} table of Figure~\ref{fig:student_update} after deleting row 3
twenty seconds after the creation of the table and the addition of
a \term{deleted} metacolumn.


\begin{figure}[h!]
\center
\begin{tabular}{ l  l  l  l  l}
id  & name      & year  & \textit{mod\_ts} & \textit{deleted} \\ 
\hline
1   & Anthony   & 4     & \textit{0}   & \textit{0}     \\
2   & Lucas     & 4     & \textit{10}   & \textit{0}     \\
3   & Steven    & 3     & \textit{20}   & \textit{1}     \\
4   & Ryan      & 4     & \textit{10}   & \textit{0}     \\
\end{tabular}
\caption{State of the \term{Students} table after marking row 3 as deleted using
the \textit{deleted} metacolumn.}
\label{fig:new_deleted}
\end{figure}

In this example, \texttt{Steven} would have been marked as deleted from the
database and hidden from the user interface. Because the change happens in a
column, it can be propagated to clients without having to change the
modification-timestamp-based approach.

Unfortunately, two additional problems arise from this fix: 
\begin{inparaenum}
\item the deleted rows build up on the server and
\item the primary keys are still occupied. 
\end{inparaenum}
The buildup of rows can be remedied by periodically removing all rows marked as
deleted with SQL \var{DELETE} queries and triggering resynchronizations on the
clients. This would cause them to delete their local databases and download all
missing rows (which would no longer include deleted rows). This is only
necessary if the deleted rows start to occupy a significant amount of space.

The primary key issue poses a greater functional threat. In Figure~\ref{fig:new_deleted}, clients
would no longer be able to add a \texttt{Steven} to the database after marking
him as deleted, nor could they modify the row to un-delete him (assuming
\texttt{name} is a primary key). To remedy this, clients are required to send a
boolean parameter called \term{replace} when making updating the server's
database through requests to \create. Figure~\ref{fig:create} depicts
the actions of the server when receiving requests to \create.


\begin{figure}[h!]
\centering
\begin{tikzpicture}[node distance = 3cm, auto, scale=.7, every node/.style={transform shape} ]
  % Place nodes
  \node [block] (request) {Request to /api/create/};
  \node [decision, below of=request] (isconflict) {Key violation?};
  
  \node [block, left of=isconflict] (create) {Create the row.};
  \node [decision, right of=isconflict] (nocreate) {Does replace=1?};
 
  \node [block, below of=create] (ok) {\textbf{200 OK}}; 

  \node [dot, below of=nocreate] (nocreateDOT) {};

  \node [block, left of=nocreateDOT] (replace) {Replace it};
  \node [decision, below of=replace] (isreplace) {Does deleted=1};

  \node [block, left of=isreplace] (error) {\textbf{409 Conflict}};

  % Draw edges
  \path [line] (request) -- (isconflict);
  \path [line] (isconflict) -- node [above] {no} (create);
  \path [line] (isconflict) -- node [above] {yes} (nocreate);
  \path [line] (create) -- (ok);

  \path [edge] (nocreate) -- (nocreateDOT);
  \path [line] (nocreateDOT) -- node [above] {yes} (replace);
  \path [line] (replace) -- (ok);

  \path [line] (nocreateDOT) |- node [right] {no} (isreplace);
  \path [line] (isreplace) -- node [right] {yes} (replace);

  \path [line] (isreplace) -- node [above] {no} (error);

\end{tikzpicture}
\caption{The actions that the server takes for every request to \create that it
receives from clients. This actions depend on clients sending a \term{replaced}
parameter with their requests.}
\label{fig:create}
\end{figure}

The flow chart starts with ``Request to /api/create/'', which is a request
to create a row on the server from a client. If there is no key constraint
violation, then the row is created and the server responds with a \texttt{200
OK}. If there is a violation then the server checks to see if the request's
\texttt{replace} parameter is set to \texttt{1}. If it is then the conflicting
row can be replaced without worry and the server responds with a \texttt{200
OK}.  If not then the server checks to see if the conflicting row is marked as
deleted. If it is, then the conflicting row can be replaced and the server
responds with a \texttt{200 OK}. If not, then the server responds with a
\texttt{409 CONFLICT} and the row is not created. At this point, the client has
enough information to ask the user if they wish to replace (or update) the
conflicting row on the server and potentially resend the request with
\texttt{replace=1}. 
 
This solution reduces the functional burden of the deleted rows, enables SQL
\texttt{UPDATE} functionality, is completely representable with pure SQL, and
requires no significant changes to the existing API. 

In summary, we have created a method of synchronization that does not utilize
and DBMS specific features or introduce any dependencies into Coexist. So far,
we have solved the problem of database synchronization, but have yet to address
the other questions presented in Section~\ref{sec:api}: what will the data
exchange format between the server and clients be, and how will clients know how
to present database contents to the user? However, we have discovered a some
requirements of client requests, namely that requests to \create must include a
\term{replace} parameter and requests to \sync must include one \term{mod\_ts}
parameter per SQL table. This information will be factored into the design of
the client-server communication model in Section~\ref{sec:communication}

%-------------------------------------------------------
%\subsection{Schema Implications}
%\label{sec:schema}
%
%% users won't have to create schemas.
%
%The synchronization depends on a few assumptions. If a row was deleted from the
%server then there would be no reliable way for the client to discover this using
%only the metacolumns. Assuming no insertions have taken place, missing rows
%would be $C - S$, where $C$ is the client side table and $S$ is the server side,
%but we would like to avoid sending the contents of entire tables over the
%network (aside from the initial synchronization). 
%
%Many of the relations of our schema include timestamps to enable us to pose more
%interesting queries. For example, our \texttt{Shelved( serial\_no, date)}
%relation allows us to tell Networkmaine all of the devices that were shelved in
%a given time range; in the past, these devices might have been deleted from
%their records entirely. For this reason, and the reason in the prior paragraph,
%we decided it would be a good idea to never delete any data from the database.
%Instead, the \texttt{deleted} metacolumn was added to each relation; a simple
%boolean toggle that tells the application that this tuple should be considered
%deleted. This enables our synchronization method and preserves the integrity of
%queries that rely on having sufficient data to operate on.
%
%Another practice to avoid is using relations to store states. In the original
%schema for Networkmaine, there was a table \hbox{\texttt{Shelved (serial, date,
%note)}} that was used to track devices that should be considered retired. These
%devices could potentially be used again and would have to be marked as deleted
%in the \texttt{Shelved} table with metacolumns. It is also reasonable to assume
%that the \texttt{Shelved} table's growth will mirror the \texttt{Device} table's
%growth because each new device will potentially replace an older device. 
%
%One possible solution would be to merge the \texttt{Shelved} table into the
%\texttt{Device} table. In MySQL, a \texttt{BOOLEAN} occupies 1 byte, a
%\texttt{TIMESTAMP} occupies 4 bytes, and \texttt{VARCHAR} occupies at least 0
%bytes and at most 3 bytes per character in utf-8 under the InnoDB engine. The
%initial database will have about six thousand devices in it, so merging the
%tables would add about $6000~bytes + 24000~bytes + 625~bytes \approx 30~kb$, or
%5 bytes per row, where the $625~bytes$ came from the sum off all of the 1 bit
%bitmasks added for allowing \texttt{NULL} values in the \texttt{VARCHAR} column;
%a negligible amount for the small growth rate of the database. The updated table
%is shown in figure~\ref{fig:updated_device}.
%
% 
%\begin{figure}[h!]
%\begin{lstlisting}[language=sql]
%CREATE TABLE Device(
%  serial VARCHAR(32) PRIMARY KEY,
%  tag VARCHAR(32) NOT NULL,
%  description VARCHAR(32),
%  shelved BOOL DEFAULT FALSE NOT NULL,
%  shelved_notes VARCHAR(32),
%  shelved_date TIMESTAMP);
%\end{lstlisting}
%\caption{Updated \texttt{Device} table.}
%\label{fig:updated_device}
%\end{figure}
%

