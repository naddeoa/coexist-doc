%--------------------------------------------------------
\section{Designing a Metamodel} \label{sec:metamodel}
%--------------------------------------------------------


%--------------------------------------------------------
\subsection{XML Meta Model}  \label{sec:xml}

A \textit{meta model} is a model of a meta data. Giulieri defined the meta data
(in this context) as ``the meaningful information necessary to fully describe
the UI and database mapping''. In this section, we analyze a representative
sample of one of Evolutility's XML meta models and explain the perceived
reasoning behind its design. Understanding this XML meta model will make the
design of the JSON version in Section~\ref{sec:json} clearer.
Figure~\ref{fig:xml} contains the XML meta model sample.

\begin{figure}[h!]
\centering

\begin{lstlisting}[language=XML]
<?xml version="1.0" encoding="UTF-8"?>
<form label="To Do"  
      xmlns="http://www.evolutility.com">
  <data entity="task" entities="tasks" 
        icon="m-todo.gif" dbtable="EVOL_ToDo" 
        dborder="PriorityID, duedate" />
  <panel label="Task" width="62" >
    <field type="text" label="Title" 
      dbcolumn="title"
      required="1" cssclass="fieldmain" 
      maxlength="255" width="100" 
      search="1" searchlist="1" searchadv="1" />
    <field type="date" label="Due Date" 
      dbcolumn="duedate" 
      maxlength="10" width="40" 
      search="1" searchlist="1" searchadv="1" />
  </panel>
</form> 
\end{lstlisting}
\caption{Sample Evolutility XML meta model}
\label{fig:xml}
\end{figure}

Looking at the XML model it is obvious that the structure is designed around the
user interface. This is noteworthy because the typical approach to
relational-database-backed applications is encumbered by the object-relational
impedance mismatch. When representing data in an object oriented language, one
must agree upon a mapping from the relational database. Technologies like the
Java Persistence API have attempted to solve this problem by allowing
programmers to implicitly specify an object mapping when creating objects in
code\cite{oracle_java_????}, but the approach presented in this section
effectively side steps the issue by mapping the database directly to UI forms,
where an obvious mapping \textit{does} exist.

With this in mind, it becomes easier to understand the semantics of the model.
The root element is a \texttt{form}, inside of which are \texttt{data} and
\texttt{panel} nodes. This information represents the structure of a \textit{ui
form}, or a section of a graphical user interface that is dedicated to user
input (like any paper form you may have encountered). The \texttt{field} nodes
are the nodes that model the actual data in the database. These are grouped
under \texttt{panel}s to represent sections of the UI, implemented, perhaps,
using tabs. Each \texttt{field} contains attributes that specify certain
database related attributes. A full list of attributes can be found on the
original Evolutility submission page\cite{giulieri_minimalist_2011}.

\begin{itemize}
\item \textbf{\texttt{dbcolumn}}: The name of the column that this field represents. The name of the table is given in the \texttt{data} node. Each form ultimately models a table in the database.
\item \textbf{\texttt{label}}: A string that will represent this field in the UI.
\item \textbf{\texttt{type}}: The type of this field. Field types are not analogous with database types.
\item \textbf{\texttt{maxlength}}: Some UI metadata to allow visual customization. 
\item \textbf{\texttt{width}}: Similar to maxlength.
\item \textbf{\texttt{search}}: Specifies whether or not this field should appear in a search form. This allows users to hide fields that get auto populated from triggers.
\end{itemize}

Using this model, Evolutility generates Web UI forms at run time. This means
that no code has to be written for the developer to configure and change the
applications functionality and appearance. 

The distinction between form types and database types is a flexible concept that
enables additional UI functionality without jeopardizing the mapping between the
database and UI. 

\begin{figure}[h!]
\centering
\begin{tikzpicture}[node distance = 2.5cm, auto, scale=.7, every node/.style={transform shape}]
  % Place nodes
  \node [block] (varchar) {varchar};
  \node [cloud, below of=varchar] (email) {email};
  \node [cloud, left of=email] (text) {text};
  \node [cloud, right of=email] (textm) {textmultiline};
  % Draw edges
  \path[line] (email) -- (varchar);
  \path[line] (text) -- (varchar);
  \path[line] (textm) -- (varchar);

\end{tikzpicture}
\caption{UI forms that map to the SQL \texttt{varchar} type.}
\label{fig:varchar_map}
\end{figure}

Figure~\ref{fig:varchar_map} shows the general relationship between UI types
(shown in red) and SQL types (shown in blue). There are generally a number of UI
types that map to a single database type. The difference between any two UI
types that map to the same database type lie in their functionality. For
example, the \texttt{text} UI type will cause fields to be presented as a single
line of text, the \texttt{textmultiline} type will cause a fields to appear as
text boxes, and the \texttt{email} type will cause fields to appear as a single
line of text and enforce email style validation. 




%--------------------------------------------------------
\subsection{JSON Metamodel} \label{sec:json}

%In this section, we examine the shortcomings of Evolutility and introduce a JSON
%model that is equivalent to the XML model of Section~\ref{sec:xml}. 
%
%
%
%\begin{figure}[h!]
%\begin{lstlisting}[language=json]
%{
%  "forms" : [
%  {
%    "label":"RMA",
%    "table":"RMA",
%    "fields": [
%      {"label":"RMA Number","type":"barcode",
%         "column":"rma_no"},
%      {"label":"Sent","type":"date",
%         "column":"sent"},
%      {"label":"Returned","type":"text",
%         "column":"returned"},
%      {"label":"Note","type":"text",
%         "column":"note"},
%      {"label":"Serial","type":"reference",
%       "column":"serial_no","required":true,
%       "references":{
%         "table":"Devices",
%         "column":"serial_no"
%    }}
%    ]
%  }
%  ]
%}
%\end{lstlisting}
%\caption{Sample JSON meta model.}
%\label{fig:json_sample}
%\end{figure}
%
%The general form of the model is similar Evolutility's XML version. The first
%attribute in the object is \texttt{form}, which references an array of forms.
%Each form contains attributes that specify the database-UI mapping and UI
%metadata. The \texttt{label} and \texttt{table} of forms represent the title and
%the database table that it models, and the \texttt{fields} array contains
%\texttt{field} objects that model the actual data. The approach to
%\texttt{fields} has been altered and expanded. In the XML model, a foreign key
%was modeled as \texttt{lov} type (list of values). It would use integer primary
%keys to identify valid values in other tables. In Coexist, that type has been
%changed to the \texttt{reference} type, and an additional \texttt{references}
%attribute has been added to model the foreign key relationship, relieving the
%integer primary key requirement. In Figure~\ref{fig:json_sample}, the column
%\texttt{serial\_no} of table \texttt{RMA} is a foreign key on the
%\texttt{serial\_no} column in the \texttt{Devices} table. 
%
%Since Coexist was designed with mobile clients in mind, there have been several
%UI types added to take advantage of smart phone functionality. The
%\texttt{barcode} type is the same as a \texttt{text} type, except it causes a
%barcode to appear on the right of it, allowing users to get input from a barcode
%using a smart phone's camera. The \texttt{image} type has been updated from
%Evolutility in this same fashion; it now presents a camera icon that allows
%users to receive input from a camera, ultimately mapping to a SQL \texttt{blob}
%type. The \texttt{location} data type allows users to use GPS as input, storing
%latitude and longitude information as a SQL \texttt{varchar} type.



%%%%%
% GOAL: be able to write metamodels by hand, being aware of all features


% before here, declare what the metamodel has to represent, and why it has 
% things like 'label' and 'table' etc.


Section~\ref{sec:replication} defined the synchronization process between the
server and clients and introduced some requirements for the communication
protocol. This section focuses on modeling the mapping of database information onto the
user interface. Specifically, this section will introduce a \textit{metamodel}
that contains all of the information that any Coexist client needs in order to
display the database contents to users. A metamodel is simply a model that deals
with \textit{metadata}, where metadata is any data about the application itself.


Without knowing any specific details about the application, we know that there
will be screens dedicated to user-data-entry. Figure~\ref{fig:sample_form}
contains a form that will appear (at least conceptually) in Coexist.

\begin{figure}[h!]
\center
\begin{tabular}{r l}
  Your name: &  \underline{\makebox[1in][1]{}}   \\
  Exercise: &  \underline{\makebox[1in][1]{}}   \\
  Weight: &  \underline{\makebox[1in][1]{}}   \\
  Date: &  \underline{\makebox[1in][1]{}}   \\
\end{tabular}
\caption{A form that is used for data entry. This simple example can be analyzed
to determine what the client needs to know about the data it handles in order to
display this form to users.}
\label{fig:sample_form}
\end{figure}

In this example, the database in question tracks statistics for workouts. The
form in Figure~\ref{fig:sample_form} allows a user to input their name, an
exercise, a weight and a date into the database. If forms of this type are going
to appear in Coexist, then clients will have to know
\begin{inparaenum}
\item what table this data should be inserted into,
\item what data goes in each column, and
\item how to filter input.
\end{inparaenum}
Figure~\ref{fig:sample_table} contains the SQL that creates the database table
that this form will upload to, without any consideration for the form's layout.

\begin{figure}[h!]
\begin{lstlisting}[language=sql]
CREATE TABLE workout_log(
  user VARCHAR(10),
  exercise VARCHAR(10) REFERENCES Exercises,
  weight INTEGER,
  date_done DATE);
\end{lstlisting}
\caption{A SQL database that tracks workout information. The form in
Figure~\ref{fig:sample_form} will be used to upload data into this table.}
\label{fig:sample_table}
\end{figure}

The ordering of the fields in the form correspond to the ordering of the columns
in the table (in this example), but the names presented in the form differ from
the column names. With this, we can start defining the contents of the Coexist
metamodel. Figure~\ref{fig:json_form} contains the JSON that will be used for
representing forms in the metamodel.

\begin{figure}[h!]
\begin{lstlisting}[language=json]
{
  "forms" : [
  {
    "label":"Workout",
    "table":"workout_log",
    "fields": [...]
  }
  ]
}
\end{lstlisting}
\caption{A sample metamodel featuring the \texttt{forms} attribute. The
\texttt{forms} attribute represents a single UI mapping for a SQL table.}
\label{fig:json_form}
\end{figure}

It's worth saying that JSON can easily be replaced with YAML or XML, as other
similar projects have done\cite{giulieri_minimalist_2011}. Coexist supports JSON
primarily, but there is nothing in the code base that specifically depends on
data being JSON -- alternative languages can be added in the future.

In Coexist, a \texttt{form} encapsulates a database table. In
Figure~\ref{fig:json_form}, the \var{forms} attribute contains an array of
\var{form} objects (only one in this case). The \var{form} object with a
\var{label} of \var{Workout} represents the database table in
Figure~\ref{fig:sample_table} and enables the form in
Figure~\ref{fig:sample_form} to display more readable names in the user
interface. The \var{fields} attribute contains an array of \var{field} objects,
which represent individual fields in forms, like the ``Name'' field in the form
of Figure~\ref{fig:sample_form}, and ultimately correspond to columns in
database tables. Figure~\ref{fig:json_field} contains the \var{field} objects that
are hidden behind the \var{\ldots} in Figure~\ref{fig:json_form}.


\begin{figure}[h!]
\begin{lstlisting}[language=json]
{
  "label":"Your name"
  "column":"user"
  "type":"string"
},
{
  "label":"Exercise"
  "column":"exercise"
  "type":"reference"
  "references":
  {
    "table":"Exercises"
    "column":"exercise"
  }

},...
\end{lstlisting}
\caption{A sample metamodel featuring the \texttt{field} object. Each
  \texttt{field} object represents a single UI mapping for a column in a SQL
table.}
\label{fig:json_field}
\end{figure}

Figure~\ref{fig:json_field} shows the UI mapping for the table's columns, where
each JSON object is a \textit{field}. The first field enables the client to
present the \var{user} column as ``Your name'' in the form. The most important
take-away from this model is the typing system, allowing the client
to filter input from users. In this example, the type of the \var{user} field is
\var{string}, which means that it accepts any string. The type of the
\var{exercise} field is \var{reference}, which is how foreign keys are modeled
in the Coexist metamodel. With forms and fields, Coexist clients can effectively
map the contents of the database onto the user interface. The next section will
define all of the types available in the Coexist metamodel.


%--------------------------------------------------------
\subsection{Metamodel types}  \label{sec:types}

Having types isn't strictly necessary -- it is possible to have clients only
display labels and entry fields and have the server reject invalid input if
it gets uploaded. Having types allows Coexist to leverage the features of
different client types effectively. For example, Coexist supports a
\var{barcode} type that allows clients on supporting platforms to read barcodes
with cameras. This section will continue to define metamodel types that Coexist
clients can use to filter user input. There are four disjoint types from which
all types inherit: \var{string}, \var{numeric}, \var{blob} and \var{reference}.
In most cases, when a subtype inherits from a parent type, the subtype may only
offer a data input hint.


%--------------------------------------------------------
\subsubsection{String types}  \label{sec:type_string}
Fields of type \var{string} can take any string as their input. They correspond
roughly to the \var{varchar} SQL type (and other types like \var{text} in different SQL
implementations). There are also several types that inherit from \var{string}.
The \var{barcode} type is treated as a string as far as a relational database is
concerned, but the Coexist client will allow users to scan barcodes when
entering data instead of having to use the keyboard. The \var{email} type is
also a subtype of \var{string}, where clients have the opportunity to validate
input as an email address before allowing requests to leave. The \var{date} and
\var{timestamp} types also inherit from \var{string}, allowing clients to
present appropriate dialogs for data entry.



%--------------------------------------------------------
\subsubsection{Numeric types}  \label{sec:type_num}

The \var{numeric} type is the base type for all numbers in Coexist. From this
type, the \var{integer} type inherits. Any data sent to the server is ultimately
sent as a string, so the differences between these types is purely cosmetic:
\var{numeric} allows numbers in $\mathbb{R}$, and \var{integer} allows numbers in
$\mathbb{Z}$. 


%--------------------------------------------------------
\subsubsection{Blob types}  \label{sec:type_blob}

Fields of type \var{blob} store some kind of blob data. The default input method
for these types is to select a file from the file system that gets dumped into
the request. For subtypes like \var{photo}, clients on supporting platforms can
launch camera applications to take pictures and send them in requests.


%--------------------------------------------------------
\subsubsection{The reference type}  \label{sec:type_reference}

The \var{reference} type is unique in that it has no subtypes. It is used to
model foreign keys in the database. It is used in Figure~\ref{fig:json_field} by
first declaring a type of \var{reference} and then by defining a
\var{references} property to let clients know which table and column the column
in question is dependent on. The \var{reference} type is a hint to clients that
the value for their field should be selected from a list of values --
specifically a list of unique attributes in the table that is referenced.

Using only the \var{field} and \var{form} JSON objects shown in
Figures~\ref{fig:json_form} and \ref{fig:json_field}, a metamodel can be
constructed that allows Coexist clients to effectively display and collect data
to and from users. Next, we consider how developers using Coexist will create
and maintain metamodels.


%--------------------------------------------------------
%\newcommand{\sql}[1]{\texttt{#1\textsubscript{sql}}}
%\newcommand{\mm}[1]{\texttt{#1\textsubscript{mm}}}
\newcommand{\sql}[1]{\texttt{$#1_{sql}$}}
\newcommand{\mm}[1]{\texttt{$#1_{mm}$}}
\subsection{Creating individual metamodels}  \label{sec:}

Having a JSON metamodel solves the problem of mapping database information onto
the user interface, but it introduces a new component that users will have some
responsibility for maintaining. Crafting small metamodels by hand is at best
time-consuming, becoming increasingly daunting as the size of database
schemas increase.  This section will introduce a program, \term{modelgen} (metamodel
generator), that mitigates this overhead for users, and discuss the reasoning
for its design.

Since the metamodel encapsulates the database schema, \term{modelgen} can either take
a metamodel and map it to SQL, or vice versa. Before choosing, we must
understand the actual mapping between the Coexist JSON metamodel and SQL,
starting with mapping SQL to the metamodel. Figure~\ref{fig:metamodel_sql}
contains three sample of semantically equivalent SQL. The first and second
samples are typical of how a human would write SQL, while the third is more
likely to come from tools like mysqldump. None the less, there is no guarantee
about which sample \term{modelgen} will be asked to convert.

\begin{figure}[h!]
\begin{lstlisting}[language=sql]
-- First sample
CREATE TABLE Enrolled (
  id INTEGER REFERENCES Students(id));

-- Second sample
CREATE TABLE Enrolled (
  id INTEGER
  CONSTRAINT fk FOREIGN KEY REFERENCES Students(id));

-- Third sample
CREATE TABLE Enrolled ( id INTEGER );
ALTER TABLE Enrolled
  ADD CONSTRAINT fk FOREIGN KEY (id) REFERENCES Students;
\end{lstlisting}
\caption{Three samples of SQL that are semantically equivalent. They will need
to be represented in the metamodel as JSON.}
\label{fig:metamodel_sql}
\end{figure}

The equivalence of these three SQL samples implies that \term{modelgen} would be a
many-to-one function if it took SQL and produced a metamodel.
Figure~\ref{fig:metamodel_json_convert1} shows the relevant portion of the
metamodel that would represent each of the three SQL samples equally.
Designing \term{modelgen} to take SQL and produce a metamodel poses several problems.

\begin{figure}[h!]
\begin{lstlisting}[language=json]
{
  "label":"Enrolled",
  "table":"Enrolled",
  "fields": [
    {"label":"Serial",
     "column":"serial_no",
     "type":"reference",
     "references":{
       "table":"Students",
       "column":"id"}}
  ]
}
\end{lstlisting}
\caption{A single field in the metamodel that equally represents all three
samples in Figure~\ref{fig:metamodel_sql}.}
\label{fig:metamodel_json_convert1}
\end{figure}


%\subsubsection{$f : type_{SQL} \rightarrow type_{metamodel}$ is not surjective}
\subsubsection{Mapping SQL types to metamodel types is not surjective}
\label{sec:metamodel_issue1}

The mapping from SQL syntax to metamodel syntax is surjective, but the
mapping from SQL types to metamodel types is not. For example, \var{id} in
Figure~\ref{fig:metamodel_sql} is an integer, but the metamodel defines
integer, and barcode, the latter of which is essentially an
integer, but allows clients to present a barcode scanner for user input (if
they support it). This means that \var{modelgen} would have to default the conversion
from SQL integer to an integer, and it would never produce a metamodel with
a barcode in it. A similar situation occurs when converting a
varchar; it can be represented as a text, email, or textbox.


%--------------------------------------------------------
\subsubsection{Metamodels represent more than SQL}  \label{sec:}

%If there are two functions, $f$ and $g$, and a metamodel, $M1$, then
%\[
%f : SQL \rightarrow metamodel
%\]
%\[
%g: metamodel \rightarrow SQL
%\]
%\[
%f(g(M1)) \neq M1
%\]

Given a function $(f:SQL \rightarrow metamodel)$ and \mbox{$(g:metamodel
\rightarrow SQL)$}, and a metamodel $M1$, $f(g(M1)) \neq M1$.
This is because the metamodel contains information that cannot be
represented in SQL, like the \var{label} field, which is used for display
purposes in client applications. Given the metamodel in
Figure~\ref{fig:metamodel_json_convert1}, if we
\begin{inparaenum}
\item change the first \var{label} property to \var{``All enrolled''},
\item convert it into SQL, and
\item convert it back into a metamodel,
\end{inparaenum}
the value of the \var{label} will be lost (defaulted to the value of the table
name), because SQL says nothing about how the data should be displayed in
applications. When converting from SQL, it is likely that a
user will be required to hand edit the metamodel and insert values for fields
like \var{label}.

In summary, the two largest issues with mapping SQL to metamodels is a lossy,
many-to-one syntax-translation and an injective, non-surjective type-translation.
Both of these issues imply that going in the reverse direction is more
favorable. There are a few stipulations, namely that 
\begin{inparaenum} 
\item one of the many possible SQL translations needs to be chosen and
\item the metamodel has to be able to represent some database related information
like primary and foreign keys.
\end{inparaenum}

Much like in Figure~\ref{fig:metamodel_sql}, we only consider different SQL
options that are semantically equivalent to each other and accurately represent
the metamodel being translated. Specifying that the SQL under consideration must
be equivalent with each other is important because it is possible to have two
different SQL statements that represent the same metamodel, but are not
technically equivalent with each other. For example, adding \var{ENGINE=InnoDB}
to sample 1 from Figure~\ref{fig:metamodel_sql} has implications on the
database, but it wouldn't change the semantics of the SQL as far as the
metamodel is concerned. To deal with this, we will only consider minimal SQL
samples, or samples of SQL that are all equivalent and don't have any additional
non-semantic parts. This usually means that we will disregard DBMS specific
features.

Since we only consider semantically equivalent SQL, any SQL chosen will work.
As a sort of tie-breaker, the SQL that makes the user's experience more
convenient will be selected. Of the three samples in Figure~\ref{fig:metamodel_sql},
sample 1 is the most simple. However, sample 2 has a significant functional advantage
over sample 1: its foreign key can be deleted because it has a name. So of the
three samples, sample 2 seems to be the best choice, and a similar reasoning
will be applied to other cases. There are several advantages to mapping
metamodels to SQL as well. Specifically, mapping in this direction allows
Coexist to do two things.


%--------------------------------------------------------
\setcounter{subsubsection}{0}
\subsubsection{Handle clients of any DBMS} \label{sec:mm_advantage}

Originally, users would have to create SQL schemas for every type of DBMS that
they intended to use on clients. If a user chose MySQL for the central server's
DBMS, and only Android clients were going to be used, then that user would have
to store a SQLite compatible version of the MySQL schema on the server so that
clients could build a local version of the central database.

If \term{modelgen} can generate schemas from metamodels then users no longer have to
store copies of SQL schemas in different DBMS-compatible forms for
clients. Figure~\ref{fig:modelgen_flow} depicts the dialog between a coexist
client and server, where the client requests the schema it should use for its
local storage.


\begin{figure}[h!]
\centering
\begin{tikzpicture}[node distance = 2.5cm, auto, scale=.7, every node/.style={transform shape}]
  % Place nodes
  \node [block] (client) {client};
  \node [block, right of=client, node distance=6cm] (server) {server};
  
  \node [dot, below of=client, label=180:1] (c1) {};
  \node [dot, below of=server] (s1) {};

  \node [dot, below of=c1, label=180:2] (c2) {};
  \node [dot, below of=s1] (s2) {};
  

  \path [line] (c1) -- node [above, align=left] {
    \tt\textbf{GET /api/sync} \\ 
    \tt db:sqlite}   (s1);


  \path [line] (s2) -- node [above, align=left] {
    \tt\textbf{200 OK} \\ 
    \tt db:<schema>}   (c2);


  \path [edge]  (client) -- node  {} (c1);
  \path [edge]  (server) -- node  {} (s1);
  
  \path [edge]  (c1) -- node [above] {} (c2);
  \path [edge]  (s1) -- node [right, align=left] {
    \tt  <schema> = \\
    \tt modelgen(model,sqlite)}   (s2);

\end{tikzpicture}
\caption{A dialog between a Coexist client and server where the client requests
the schema it should use for local storage.}
\label{fig:modelgen_flow}
\end{figure}

Figure~\ref{fig:modelgen_flow} starts at step 1, where the client requests the
schema it should be using for its local storage. The server generates the
schema in a SQLite compatible form and returns it to the client in step 2. This
behavior requires no changes to the communication model established in
Section~\ref{sec:communication}.




%--------------------------------------------------------
\subsubsection{Add metacolumns to the schema} \label{sec:mm_advantage}

As discussed in Section~\ref{sec:replication}, database schemas require the
addition of metamodels to enable database replication. If \term{modelgen} is creating
the schemas, then the metacolumns, triggers, and indexes can be automatically
added. Figure~\ref{fig:modelgen_save} shows the non-trivial amount of SQL
boilerplate that users would have to add themselves for \emph{every} table
in their database schema.


\begin{figure}[h!]
\begin{lstlisting}[language=SQL]
-- Adding two metacolumns to tables
CREATE TABLE Name (
  ...,
  mod_ts TIMESTAMP,
  deleted INTEGER);
-- A trigger for UPDATE, INSERT, and REPLACE
CREATE TRIGGER mod_ts AFTER INSERT ON Student
  FOR EACH ROW SET mod_ts = NOW();
-- An index on the commonly searched mod_ts
CREATE INDEX modts_index ON Name (mod_ts DESC);

\end{lstlisting}
\caption{Three types of SQL statements that need to be added to every table in
every schema in Coexist. These statements are automatically added when  users
generate schemas with \term{modelgen}.}
\label{fig:modelgen_save}
\end{figure}


At this point, we have addressed all of the concerns in Section~\ref{sec:api}.
We have defined a pure SQL synchronization algorithm in
Section~\ref{sec:replication}, outlined a communication protocol in
Section~\ref{sec:communication}, and specified a JSON based metamodel that
clients can use to map database contents onto user interfaces in this section.
Next, in Section~\ref{sec:deployment}, we examine the Coexist system as a whole,
from a user's point of view.
%Next, in Section~\ref{sec:api_imp}, we will revisit the API methods declared in
%Section~\ref{sec:api} and finalize their implementation. 



